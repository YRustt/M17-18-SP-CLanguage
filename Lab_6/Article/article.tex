\documentclass[12pt]{article}
\usepackage[left=1.5cm,right=1.5cm, top=1.5cm,bottom=1.5cm,bindingoffset=0cm]{geometry}
\usepackage[T2A]{fontenc}
\usepackage[utf8]{inputenc}
\usepackage[russian]{babel}
\usepackage{listings}


\begin{document}
\section{C-расширения}

\subsection{Часть 1. ctypes}

В \textbf{sum1.c} запишем функцию для подсчёта суммы цифр числа:
\begin{lstlisting}[language=C, xleftmargin=\parindent, frame=L, numbers=left]
unsigned digit_sum(unsigned x) {
    unsigned ans = 0;
    while (x != 0) {
        ans += x % 10;
        x /= 10;
    }
    return ans;
}
\end{lstlisting}
Соберём $shared\ library$ из нашего исходника следующим образом:
\begin{lstlisting}[language=bash, xleftmargin=\parindent]
gcc -O3 -shared -o sum1.so -fPIC sum1.c
\end{lstlisting}
В \textbf{py1.py} при помощи библиотеки $ctypes$ загрузим нашу $shared\ library$ и посчитаем сумму цифр чисел от $0$ до $100$ млн.:
\begin{lstlisting}[language=Python, xleftmargin=\parindent, frame=L, numbers=left]
from ctypes import CDLL


libc = CDLL("sum1.so")
digit_sum = libc.digit_sum


res = 0
for i in range(100000000):
    res += digit_sum(i)

print(res)
\end{lstlisting}
Запустим и посмотрим на результаты:
\begin{lstlisting}
yrustt@yrustt:~/Desktop/Projects/M17-18-SPCL/Lab_6$ time python3 py1.py 
3600000000

real    0m44.089s
user    0m44.064s
sys     0m0.004s
\end{lstlisting}
\newpage
\subsection{Часть 2. CPython Extension Module}

В \textbf{sum2module.c} определим функцию $digit\_sum$. 
\newline
\newline
Также в этом файле определяем массив структур $PyMethodDef$ - функции, которые могут вызываться из $python$-кода. Флаг $METH\_VARARGS$ указывает, что мы не будем передавать $keyword$ аргументы в функцию при вызове, что позволяет нам воспользоваться функцией $PyArg\_ParseTuple$.
\newline
\newline
Также определим структуру $PyModuleDef$.
\begin{lstlisting}[language=C, xleftmargin=\parindent, frame=L, numbers=left]
static PyObject* digit_sum(PyObject *self, PyObject *args) {
    int x, ans = 0;
    if (!PyArg_ParseTuple(args, "i", &x)) return NULL;

    while (x != 0) {
        ans += x % 10;
        x /= 10;
    }

    return Py_BuildValue("i", ans);
}
...
\end{lstlisting}
В \textbf{setup.py} определим:
\begin{lstlisting}[language=Python, xleftmargin=\parindent, frame=L, numbers=left]
from distutils.core import setup, Extension

setup(
    name='sum2',
    version='0.0.1',
    ext_modules=[Extension('sum2', ['sum2module.c'])]
)
\end{lstlisting}
Чтобы получить $shared\ library$ из нашего $c$-кода, запускаем:
\begin{lstlisting}[language=bash, xleftmargin=\parindent]
python3 setup.py install
\end{lstlisting}
В \textbf{py2.py} импортируем наш модуль и посчитаем сумму цифр чисел от $0$ до $100$ млн.:
\begin{lstlisting}[language=Python, xleftmargin=\parindent, frame=L, numbers=left]
import sum2

res = 0
for i in range(100000000):
    res += sum2.digit_sum(i)

print(res)
\end{lstlisting}
Запустим и посмотрим на результаты:
\begin{lstlisting}
yrustt@yrustt:~/Desktop/Projects/M17-18-SPCL/Lab_7$ time python3 py2.py 
3600000000

real    0m29.391s
user    0m29.388s
sys     0m0.000s
\end{lstlisting}
\newpage
\subsection{Бенчмарк}
\begin{tabular}{|l|r|}
\hline
\textbf{Method} & \textbf{Time (s)} \\
\hline
Python & 150.192 \\
\hline
ctypes & 44.089 \\
\hline
CPython Extension Module & 29.391 \\
\hline
\end{tabular}
\end{document}